\documentclass[12pt, a4paper, final, factor=1100, stretch=16, shrink=16]{moderncv}

\usepackage{color}
\usepackage{fontspec}

% Stop bugging me about \es
\usepackage{xspace}
% moderncv themes
\moderncvstyle{classic}                             % style options are 'casual' (default), 'classic', 'banking', 'oldstyle' and 'fancy'
%\moderncvcolor{blue}                               % color options 'black', 'blue' (default), 'burgundy', 'green', 'grey', 'orange', 'purple' and 'red'

% the (custom) color which will be used in the cv
\definecolor{color1}{RGB}{100, 14, 39}

% scale the page layout (depreciated for more complicated stuff)
%\usepackage[scale=0.75]{geometry}

% change width of the column with the dates
\setlength{\hintscolumnwidth}{2.5cm} 

%\PassOptionsToPackage[final, factor=1100, stretch=18, shrink=18 ]{microtype} 	
% The final option overrides global defaults. It greatly improves general appearance of the text. The stretch and shrink reduce bluriness[20,20 default]. The factor increases protrusion amount by 10%, [default 1000]
% Tracking allows for small caps, like in cites to be adjusted. The activate commands are to set protrusion.

\microtypecontext{spacing=nonfrench} 						% To preserve interword spacing via \nonfrenchspacing.
 
\SetExtraKerning[unit=space] 								% These produce more effects, from microtype with kerning.
    {encoding={*}, family={bch}, series={*}, size={footnotesize,small,normalsize}}
    {\textendash={400,400}, 								% en-dash, add more space around it
     "28={ ,150}, 											% Left bracket, add space from right
     "29={150, },											% Right bracket, add space from left
     \textquotedblleft={ ,150}, 							% Left quotation mark, space from right
     \textquotedblright={150, }} 							% Right quotation mark, space from left


\SetExtraKerning[unit=space]
  {encoding={*}, family={qhv}, series={b}, size={large,Large}}
  {1={-200,-200}, 
   \textendash={400,400}}

\SetTracking{encoding={*}, shape=sc}{40} 					% This is for better small caps with tracking and microtype.

\SetProtrusion{encoding={*},family={bch},series={*},size={6,7}}  % This enables better optical views.
              {1={ ,750},2={ ,500},3={ ,500},4={ ,500},5={ ,500},
               6={ ,500},7={ ,600},8={ ,500},9={ ,500},0={ ,500}}

% Page settings
\usepackage{geometry}
 \geometry{
 a4paper,
 %total={210mm,297mm},
 left=10mm,
 right=10mm,
 % top=10mm,
 % bottom=08mm,
 }

% required when changing page layout lengths
\AtBeginDocument{\recomputelengths} 
\usepackage{xunicode}
\usepackage{xltxtra}
% \usepackage[utf8]{inputenc} (Xelatex expects UTF8 anyway)

% I like pretty logos
\usepackage{metalogo}

% Widen the letter spacing for logos
\setlogokern{La}{0.2pt}
\setlogokern{Xe}{0.2pt}
\setlogokern{eL}{0.2pt}
\setlogokern{Te}{0.2pt}
\setlogokern{aT}{0.2pt}
\setlogokern{eX}{0.2pt}

% Better Quotes (depreciated for this useage)
%\usepackage{epigraph}

% Poaching from Espresso's ug.tex

%%%%%%%%%%%%%%%%%%%%%%%%%%%%%%%%%%%%%%%%%%%%%%%%%%
%%%%%%%%%%%%%%%%%%%%%%%%%%%%%%%%%%%%%%%%%%%%%%%%%%
%%%%%%%%% New Commands and Environments %%%%%%%%%%
%%%%%%%%%%%%%%%%%%%%%%%%%%%%%%%%%%%%%%%%%%%%%%%%%%
%%%%%%%%%%%%%%%%%%%%%%%%%%%%%%%%%%%%%%%%%%%%%%%%%%
\newcommand{\es}{\mbox{\textsf{ESPResSo}}\xspace}

% german word break/hyphenation rules that's with ngerman (we use english)
\usepackage[british,UKenglish,USenglish,american]{babel}

% insert dummy text (used in the letter)
%\usepackage{lipsum} 

% used for \begin{comment}...\end{comment}
\usepackage{verbatim} 

% use guilllemets in bibliography (not german.) Also autostyle superceeded babel
\usepackage[autostyle=true]{csquotes}

\usepackage[
	sorting=none,
	minbibnames=8,
	maxbibnames=9,
	block=space,
  defernumbers=true
]{biblatex}

\bibliography{publications}


% get rid of the number-labels ([1], [2], etc.) in front of publications
\defbibenvironment{midbib}
{\list
	{}
	{
		\setlength{\leftmargin}{0mm}
		\setlength{\itemindent}{-\leftmargin}
		\setlength{\itemsep}{\bibitemsep}
		\setlength{\parsep}{\bibparsep}}
	}
	{\endlist}
{\item}


% add [DOI] and [PDF] fields at the end of each publication entry
\DeclareSourcemap{
	\maps[datatype=bibtex]{
		% the bibtex entry 'mydoi' gets mapped to 'usera'
		\map{
			\step[fieldsource=mydoi]
			\step[fieldset=usera, origfieldval]
		}
		% the bibtex entry 'mypdf' gets mapped to 'usera'
		\map{
			\step[fieldsource=mypdf]
			\step[fieldset=userb, origfieldval]
		}
	}
}

% [DOI] entries in publication
\DeclareFieldFormat{usera}{\color{color1}[\href{#1}{\textsc{doi}}]}
\AtEveryBibitem{
	% put [DOI] stuff at the end of a publication entry
	\csappto{blx@bbx@\thefield{entrytype}}{% 
		\iffieldundef{usera}{ 
			% this gets invoked, once nothing is supplied
			% via the mypdf or mydoi value.
			% you could e.g. display a default thing here.
		}{\space\printfield{usera}}}
}

% [PDF] entries in publication
\DeclareFieldFormat{userb}{\color{color1}[\href{#1}{\textsc{pdf}}]}

\AtEveryBibitem{
	% put [DOI] stuff at the end of a publication entry
	\csappto{blx@bbx@\thefield{entrytype}}{\iffieldundef{userb}{}{\printfield{userb}}}
}

\renewcommand*{\mkbibnamegiven}[1]{%
	\ifitemannotation{highlight}
	{\textbf{#1}}
	{#1}}

	\renewcommand*{\mkbibnamefamily}[1]{%
	\ifitemannotation{highlight}
	{\textbf{#1}}
	{#1}
}


% Minion Pro is used as the main font, if you don't
% have it installed uncomment this line or choose another pretty font.
% Literata is also included.
%\setmainfont[Numbers=OldStyle]{Minion Pro}

\setmainfont[
Path = Fonts/MinionPro/,
Numbers=OldStyle,
Extension = .otf,
UprightFont = *_Regular,
ItalicFont = *_It,
BoldFont = *_Bold,
BoldItalicFont = *_BoldIt
]{MinionPro}

% Myriad Pro is used as the sans font, if you don't
% have it installed uncomment this line or choose another pretty font.
\setsansfont[
Path = Fonts/MyriadPro/,
Numbers=OldStyle,
Extension = .otf,
UprightFont = *_Regular,
ItalicFont = *_It,
BoldFont = *_Bold,
BoldItalicFont = *_BoldIt,
Ligatures=TeX,
Scale=MatchLowercase]{MyriadPro}

% the moderncv package will populate a lot of the pdf meta-information.
% this can be used to change some of these infos.
\AfterPreamble{\hypersetup{
  pdfcreator={XeLaTeX},
  pdftitle={First fledging CV}
}}

% for the icons (telephone, globe). I found the icons provided by the
% fontawesome package prettier than the standard moderncv icons.
\defaultfontfeatures{
    Path = Fonts/FontAwesome/ }

% personal data
\firstname{Ruhila}
\familyname{S.}
\address{Hostel 7}{IISER Mohali, 140306,}
\quote{``Nothing exists for itself alone, but only in relation to other forms of life.''\\-- Charles Darwin}

% \faEnvelope \faPhone \faGithub \faGlobe

% I use the extrainfo command for additional information, since I 
% want to use custom icons and have finer control over spacing.
\extrainfo{
	{\small\faEnvelope}\hspace{0.3em}ms20110@iisermohali.ac.in\\
	{\small\faGithub}\hspace{0.3em}RuhilaS\\
	\faGlobe\hspace{0.3em}https://github.com/RuhilaS/
}


% picture, resized to a height of 84pt
\photo[84pt]{Picture/ruhi}

% spacing before (sub)sections
\newcommand{\spacesection}{\vspace{0.4cm}}
\newcommand{\spacesubsection}{\vspace{0.2cm}}


%===========================
\begin{document}

% Adapted from https://tex.stackexchange.com/a/170219/130845
\begin{tikzpicture}[remember picture,overlay]
      \node[anchor=north, yshift=-0.25cm] at (current page.north) {\underline{Last
		  updated on \today}};
\end{tikzpicture}

\maketitle

\section{Personal Data}
\cvitem{Name}{Ruhila S.}
\cvitem{Date Of Birth}{20.09.2001}
\cvitem{Birthplace}{Virudhunagar, Tamil Nadu, India}

\spacesection{}
\section{Education}
\cventry{\textsc{2021--present}}{B.S-M.S. Biology (Major) Data Science (Minor)}{Indian Institute of Science Education and Research (IISER)}{Mohali}{\emph{India}}{$8.3$ CGPA}
% (\textsc{Advisor: } Prof. ; \textsc{Co-Advisor: } Prof. )}
%\textsc{Project: } )}
\cventry{\textsc{2020}}{Intermediate (AISSCE)}{Velammal Vidyalaya, Ayanabakkam}{Chennai}{\emph{India}}{$83.8\%$ Central Board of Secondary Education (CBSE)}
\cventry{\textsc{2017}}{High School (AISSE)}{Velammal Vidyalaya, Ayanabakkam}{Chennai}{\emph{India}}{$10.0$ Cumulative Grade Point Average (CGPA) in Central Board of Secondary Education (CBSE)}

\spacesection
\section{Experience}
\subsection{Internships}

\cventry{\textsc{Winter 2022}}{Dr. Nagma Parveen}{IIT Kanpur}{}{Research Intern}{I shall be working on wet-lab methods for studying the mechanisms of viral action under varying external stimuli.}

\cventry{\textsc{Summer 2022--present}}{Prof. Arnar Pálsson}{University of Iceland}{}{Research Staff}{Detailed analysis in a literate and reproducible manner for simulating a series of possible molecular evolutionary pathways for the \textit{Salmolid} using phylogenetic trees. This involved the five steps on an HPC (High Performance Computer) with literate programming visualization in R:
  \begin{itemize}
          \item Data curation with NCBI databases
          \item Homology inference using similarity measures (BLAST)
          \item Multiple sequence alignment (MUSCLE5)
          \item Alignment trimming (G-BLOCK)
          \item Tree simulation with distance measures (BIONJ) and maximum likelihood approaches (RAXML-NG)
  \end{itemize}~\\
 \textsc{Project Report: }Computational Primitives for Evolutionary Paths ($\approx 147$ pages)
}

\cventry{\textsc{Summer 2021}}{Dr. Lolitika Mandal}{IISER Mohali}{}{Research Intern}{Exploring Genetic Tools for working with Drosophila from a wet-lab perspective of data collection and analysis.}

\subsection{Volunteer Work}

\cventry{\textsc{2022--present}}{IEEE P3173}{IEEE Standards Committee}{}{Secretary}{Am supporting the drafting the IEEE Standard for Endocrine Disrupting Chemical Hazard Labeling}

\cventry{\textsc{2021--present}}{Biological Society}{IISER Mohali}{}{Member}{Enthusiastic participant and also am responsible for arranging independent peer-reviewed article readings.}

\cventry{\textsc{2021--present}}{Dance Society}{IISER Mohali}{}{Member}{Active participant for choreography and performances.}

\spacesection{}
\section{Certifications}

% Consider moving to be below Education
\spacesubsection{}
\subsection{NPTEL Courses}
\cventry{\textsc{Sep 2022}}{Applications of machine learning techniques in biology using WEKA}{IIT Madras}{Distinction}{$93\%$}{} % TODO: Add certificate

\subsection{Other Courses}
\cventry{\textsc{Nov 2022}}{Practical Python for beginners: a biochemist's guide}{Biochemical Society, U.K.}{}{}{}
\cventry{\textsc{Nov 2022}}{The future of HPC programming - a Modern Fortran workshop}{Swedish National Infrastructure for Computing, Online}{}{}{}

\section{Technical Skills}
\subsection{Programming Languages}
\spacesubsection{}

\cvdoubleitem{\textsc{Experienced}}{R, Python (3.x), Shell (zsh,bash)}
           {\textsc{Familiar}}{C, Java}

% \newpage
\spacesubsection{}
\subsection{Bioinformatics Packages}
\spacesubsection{}

\cvdoubleitem{\textsc{Experienced}}{Randomized Axelerated Maximum Likelihood new generation (RAXML-NG), MUSCLE5 (multiple sequence alignment)}
           {\textsc{Familiar}}{WEKA, BEAST2 (Bayesian Evolutionary Analysis Sampling Trees) via babette, Snakemake}

\spacesubsection{}
\subsection{Tools}
\spacesubsection{}

\cvdoubleitem{\textsc{Experienced}}{Git (version control), ssh, Vim, Markdown}
           {\textsc{Familiar}}{Office-Suites (MS, OpenOffice, LibreOffice)}

\spacesubsection{}
\subsection{Experimental}
\spacesubsection{}

\cvdoubleitem{\textsc{Biological}}{Handling flies (wild-type, w118, tubby), Drosophila larva dissection (brain, salivary gland, proventriculus, imaginal discs, gastric caeca), Fixing, staining, mounting viewing tissues with Flourescent microscopes, Making PBS, PFA}
           {\textsc{Professional}}{Time management, critical thinking, problem solving, communication}

% \newpage
% \spacesection
\section{Research Topics}
\cvdoubleitem{\textsc{Experienced}}{Phylogenetic Tree Construction (Distance, Maximum Likelihood, Bayesian), Evolutionary Biology, Population genetics, R reproducible literate programming, High performance open source software, Scientific Software Development for High Throughput calculations}
           {\textsc{Interested}}{Biomolecular simulations, Genomics, Modeling genetic markers for disease, Oncology and stem cells, Human genetics}

\spacesection{}
\section{Affiliations}
\subsection{Memberships}
\spacesubsection{}
\cventry{\textsc{2022--present}}{RSC (Royal Society of Chemistry)}{}{Student Member}{}{}
\cventry{\textsc{2022--present}}{RSB (Royal Society of Biology)}{}{Student Member}{}{}
\cventry{\textsc{2022--present}}{British Ecological Society}{}{Student Member}{}{}
\cventry{\textsc{2022--present}}{Biochemical Society, UK}{}{Undergraduate Member}{}{}
\cventry{\textsc{2022--present}}{Genetics Society, UK}{}{Undergraduate Member}{}{}
\cventry{\textsc{2022--present}}{Genetics Society of America}{}{Undergraduate Member}{}{}
\cventry{\textsc{2022--present}}{Royal Microscopical Society, UK}{}{Undergraduate Member}{}{}
\cventry{\textsc{2022--present}}{IEEE EMBS (Engineering in Medicine and Biology Society)}{}{Student Member}{}{}
\cventry{\textsc{2022--present}}{Federation of European Biochemical Societies (FEBS)}{}{Member}{}{}
\cventry{\textsc{2022--present}}{European Microscopy Society}{}{Member}{}{}
\cventry{\textsc{2022--present}}{ACM (Association for Computing Machinery)}{}{Student Member}{}{}

\spacesection{}
\section{Publications}
%  {\color{color1}\textsc{Journals}}

% \nocite{*}
% \printbibliography[heading=none, env=midbib, keyword=journal]

{\color{color1}\textsc{Conference Proceedings}}

\nocite{*}
\printbibliography[heading=none, env=midbib, keyword=conference]

% Currently none which haven't been published
% {\color{color1}\textsc{Preprints}}

% \nocite{*}
% \printbibliography[heading=none, env=midbib, keyword=preprint]

\spacesection{}
\section{Conference Records}

\subsection{Posters}
\cventry{\textsc{November 2022}}{Tracing Lineages of \textit{Salmo Salar} through Histone sequence data}{BES Annual Meeting 2022}{\underline{Ruhila S.}}{Accepted}{}

\subsection{Oral Presentations}
\cventry{\textsc{November 2022}}{High Throughput Reproducible Literate Phylogenetic Analysis}{IEEE PDGC-2022}{R. Goswami, \underline{Ruhila S.}}{Accepted}{}
\cventry{\textsc{November 2022}}{Reproducible High Performance Computing Without Redundancy with Nix}{IEEE PDGC-2022}{R. Goswami, \underline{Ruhila S.}, A. Goswami, S. Goswami and D. Goswami}{Accepted}{}
\cventry{\textsc{November 2022}}{Reproducible Literate Programming Workflows for Censored Data}{IOP Machine Learning for Healthcare}{R. Goswami, \underline{R. S.}}{}{}

%=============================
% this part is a simple cover letter
%\clearpage

%\recipient{Human Resources}{Some Company Ltd.\\Some Street 123\\12345 Some City} 
%\date{\today}
%\opening{Dear Sir or Madam,}
%\closing{Sincerely yours,}
%\enclosure[Attached]{curriculum vit\ae{}}

%\makelettertitle

%\lipsum[1-3] 

%\makeletterclosing


\end{document}

%%% Local Variables:
%%% mode: latex
%%% TeX-master: t
%%% End:
